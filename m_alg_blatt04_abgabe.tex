% ----------------------- TODO ---------------------------
%Template 
\documentclass[a4paper]{scrartcl}
\usepackage[utf8]{inputenc}
%\usepackage[ngerman]{babel}
\usepackage{geometry,forloop,fancyhdr,fancybox,lastpage}
\usepackage{listings}
\lstset{frame=tb,
	language=Java,
	aboveskip=3mm,
	belowskip=3mm,
	showstringspaces=false,
	columns=flexible,
	basicstyle={\small\ttfamily},
	numbers=left,
	numberstyle=\tiny\color{gray},
	keywordstyle=\color{blue},
	commentstyle=\color{dkgreen},
	stringstyle=\color{mauve},
	breaklines=true,
	breakatwhitespace=true,
	tabsize=3
}
\geometry{a4paper,left=3cm, right=3cm, top=3cm, bottom=3cm}
% Diese Daten müssen pro Blatt angepasst werden:
\newcommand{\NUMBER}{4}
\usepackage{graphicx}
\graphicspath{{/home/stefan/projects/m_alg/}}
\newcommand{\EXERCISES}{4}
% Diese Daten müssen einmalig pro Vorlesung angepasst werden:
\newcommand{\COURSE}{Methoden der Algorithmik}
%\newcommand{\TUTOR}{Benjamin Coban}
\newcommand{\STUDENTA}{Stefan Wezel}
\newcommand{\STUDENTB}{Lukas Günthner}
%\newcommand{\STUDENTC}{Gwent Krause}
\newcommand{\DEADLINE}{\date}
% ----------------------- TODO ---------------------------



%Math
\usepackage{amsmath,amssymb,tabularx}

%Figures
\usepackage{graphicx,tikz,color,float}
\graphicspath{ {home/stefan/picures/} }
\usetikzlibrary{shapes,trees}

%Algorithms
\usepackage[ruled,linesnumbered]{algorithm2e}

%Kopf- und Fußzeile
\pagestyle {fancy}
%\fancyhead[L]{Tutor: \TUTOR}
\fancyhead[C]{\COURSE}
\fancyhead[R]{\today}

\fancyfoot[L]{}
\fancyfoot[C]{}
\fancyfoot[R]{Seite \thepage}

%Formatierung der Überschrift, hier nichts ändern
\def\header#1#2{
	\begin{center}
		{\Large\bf Übungsblatt #1}\\
		{(Abgabetermin #2)}
	\end{center}
}

%Definition der Punktetabelle, hier nichts ändern
\newcounter{punktelistectr}
\newcounter{punkte}
\newcommand{\punkteliste}[2]{%
	\setcounter{punkte}{#2}%
	\addtocounter{punkte}{-#1}%
	\stepcounter{punkte}%<-- also punkte = m-n+1 = Anzahl Spalten[1]
	\begin{center}%
		\begin{tabularx}{\linewidth}[]{@{}*{\thepunkte}{>{\centering\arraybackslash} X|}@{}>{\centering\arraybackslash}X}
			\forloop{punktelistectr}{#1}{\value{punktelistectr} < #2 } %
			{%
				\thepunktelistectr &
			}
			#2 &  $\Sigma$ \\
			\hline
			\forloop{punktelistectr}{#1}{\value{punktelistectr} < #2 } %
			{%
				&
			} &\\
			\forloop{punktelistectr}{#1}{\value{punktelistectr} < #2 } %
			{%
				&
			} &\\
		\end{tabularx}
	\end{center}
}

\begin{document}
	
	\begin{tabularx}{\linewidth}{m{0.2 \linewidth}X}
		\begin{minipage}{\linewidth}
			\STUDENTA\\
			\STUDENTB\\
			%\STUDENTC
		\end{minipage} & \begin{minipage}{\linewidth}
			\punkteliste{1}{\EXERCISES}
		\end{minipage}\\
	\end{tabularx}
	
	%\header{Nr. \NUMBER}{\DEADLINE}
	
	% ----------------------- TODO ---------------------------
	% Hier werden die Aufgaben/Lösungen eingetragen:
	










































\section*{Aufgabe 3}
\subsection*{(a)}
Ist $m$ die Anzahl an verschiedenen Motiven, ergibt sich folgende Formel zur Berchnung.\\

$\frac{m \cdot \sum_{i=1}^{m}\frac{1}{i}}{24}$\\

Wir nehmen dabei an, dass ein Adventskalender $24$ Türchen hat. Kaufen wir nun einen Adventskalender, erhalten wir also $24$ Motive auf einmal, wobei diese natürlich doppelt vorkommen können. Die Summe im Zähler entsteht durch die Wahrscheinlichkeit einzelne Motive zu treffen und beschreibt deren Erwartungswert. Unser gesamter Nenner ist dann der Erwartungswert, wie viele einzelne Türchen wir öffnen müssen, um alle $m$ Motive zu bekommen. Da wir mit jedem Kalender $24$ bekommen, müssen wir unseren Erwartungswert also noch durch $24$ teilen.


\subsection*{(b)}
\subsubsection*{m=10:}
$\frac{10 \cdot \sum_{i=1}^{10}\frac{1}{i}}{24} \approx 1.2$

\subsubsection*{m=20:}
$\frac{20 \cdot \sum_{i=1}^{20}\frac{1}{i}}{24} \approx 3$

\subsubsection*{m=30:}
$\frac{30 \cdot \sum_{i=1}^{30}\frac{1}{i}}{24} \approx 5$

\subsubsection*{m=40:}
$\frac{40 \cdot \sum_{i=1}^{40}\frac{1}{i}}{24} \approx 7.1$



\subsection*{(c)}
$
\frac{m \cdot \sum_{i=1}^{m}\frac{1}{i}}{24} = 24
$\\
Nun müssen wir also einen Wert für die Summe finden, dass im Zähler 576 steht, um die Gleichung zu erfüllen.\\
$\Rightarrow \sum = \frac{576}{m}$\\
Da es sich bei der Summe um die Harmonische Reihe $\mathbb{H}$ handelt können wir diese betrachten und finden heraus, dass gilt $\mathbb{H}_{110} = \frac{576}{110}$.\\
Damit man 24 Adventskalender kaufen muss, um alle Motive zu erhalten, müsste die Firma also $110$ verschiedene Motive herstellen.


\section*{Aufgabe 4}
\subsection*{(a)}
$x$ ist so gewählt, dass es immer einen Wert im Intervall $[0,1]$ ergibt, welche den Abstand zwischen den Werten darstellen. Wenn nun gilt $n \rightarrow \infty$, summieren diese Werte unendlich oft auf, und bekommen so letztendlich die Fläche unter der Funktion, was ja das Integral ist.

\begin{figure}[h]
	\includegraphics[scale=.7]{monte_carlo_integration.png}
	\caption{Für $n \rightarrow \infty$ werden die einzelnen (grün markierten) Flächen unendlich klein, und es existieren unendlich viele, dadurch entspricht die Summe dieser Flächen exakt dem Integral. Wählt man $n \leq \infty$ handelt es sich lediglich um eine Approximation des Integrals.}
\end{figure}


\subsection*{(b)}
Da wir von einer uniformen Verteilung ausgehen, können wir die Wahrscheinlichkeit einen bestimmten Punkt zu treffen durch $\frac{1}{n}$ berechnen.\\ 
Summieren wir nun diese Wahrscheinlichkeiten auf ergibt sich offensichtlich $\frac{1}{n} \cdot \sum_{}^{n}$, was genau unserer Formel entspricht.



\subsection*{(c)}b
\begin{lstlisting}
function integral(number_of_tests, f):
	init integral = 0
	random_number = get_random_number_uniform(0,1)
	for i in range(number_of_tests):
		y = f(random_number)
		integral += y / number_of_tests
	return integral
\end{lstlisting}
Die Funktion hat eine lineare Laufzeit, da sie genau eine for-Schleife mit $n = number\_of\_tests$ Iterationen beinhaltet. Da sowohl $f$ als auch $get_random_number_uniform$ in konstanter Zeit ablaufen, und sonst lediglich Variablenzuweisungen und eine Summation passieren, welche ebenfalls in konstanter Zeit ablaufen passieren, bleibt die Laufzeit linear.







\end{document}