% ----------------------- TODO ---------------------------
%Template 
\documentclass[a4paper]{scrartcl}
\usepackage[utf8]{inputenc}
%\usepackage[ngerman]{babel}
\usepackage{geometry,forloop,fancyhdr,fancybox,lastpage}
\usepackage{listings}
\lstset{frame=tb,
	language=Java,
	aboveskip=3mm,
	belowskip=3mm,
	showstringspaces=false,
	columns=flexible,
	basicstyle={\small\ttfamily},
	numbers=left,
	numberstyle=\tiny\color{gray},
	keywordstyle=\color{blue},
	commentstyle=\color{dkgreen},
	stringstyle=\color{mauve},
	breaklines=true,
	breakatwhitespace=true,
	tabsize=3
}
\geometry{a4paper,left=3cm, right=3cm, top=3cm, bottom=3cm}
% Diese Daten müssen pro Blatt angepasst werden:
\newcommand{\NUMBER}{4}
\usepackage{graphicx}
%\graphicspath{{maps/}}
\newcommand{\EXERCISES}{4}
% Diese Daten müssen einmalig pro Vorlesung angepasst werden:
\newcommand{\COURSE}{Methoden der Algorithmik}
%\newcommand{\TUTOR}{Benjamin Coban}
\newcommand{\STUDENTA}{Stefan Wezel}
\newcommand{\STUDENTB}{Lukas Günthner}
%\newcommand{\STUDENTC}{Gwent Krause}
\newcommand{\DEADLINE}{\date}
% ----------------------- TODO ---------------------------



%Math
\usepackage{amsmath,amssymb,tabularx}

%Figures
\usepackage{graphicx,tikz,color,float}
\graphicspath{ {home/stefan/picures/} }
\usetikzlibrary{shapes,trees}

%Algorithms
\usepackage[ruled,linesnumbered]{algorithm2e}

%Kopf- und Fußzeile
\pagestyle {fancy}
%\fancyhead[L]{Tutor: \TUTOR}
\fancyhead[C]{\COURSE}
\fancyhead[R]{\today}

\fancyfoot[L]{}
\fancyfoot[C]{}
\fancyfoot[R]{Seite \thepage}

%Formatierung der Überschrift, hier nichts ändern
\def\header#1#2{
	\begin{center}
		{\Large\bf Übungsblatt #1}\\
		{(Abgabetermin #2)}
	\end{center}
}

%Definition der Punktetabelle, hier nichts ändern
\newcounter{punktelistectr}
\newcounter{punkte}
\newcommand{\punkteliste}[2]{%
	\setcounter{punkte}{#2}%
	\addtocounter{punkte}{-#1}%
	\stepcounter{punkte}%<-- also punkte = m-n+1 = Anzahl Spalten[1]
	\begin{center}%
		\begin{tabularx}{\linewidth}[]{@{}*{\thepunkte}{>{\centering\arraybackslash} X|}@{}>{\centering\arraybackslash}X}
			\forloop{punktelistectr}{#1}{\value{punktelistectr} < #2 } %
			{%
				\thepunktelistectr &
			}
			#2 &  $\Sigma$ \\
			\hline
			\forloop{punktelistectr}{#1}{\value{punktelistectr} < #2 } %
			{%
				&
			} &\\
			\forloop{punktelistectr}{#1}{\value{punktelistectr} < #2 } %
			{%
				&
			} &\\
		\end{tabularx}
	\end{center}
}

\begin{document}
	
	\begin{tabularx}{\linewidth}{m{0.2 \linewidth}X}
		\begin{minipage}{\linewidth}
			\STUDENTA\\
			\STUDENTB\\
			%\STUDENTC
		\end{minipage} & \begin{minipage}{\linewidth}
			\punkteliste{1}{\EXERCISES}
		\end{minipage}\\
	\end{tabularx}
	
	%\header{Nr. \NUMBER}{\DEADLINE}
	
	% ----------------------- TODO ---------------------------
	% Hier werden die Aufgaben/Lösungen eingetragen:
	
	\section*{Aufgabe 1}
	
	
	
		\includegraphics[scale=0.4]{/home/stefan/projects/m_alg/maps/blatt2_img/aufgabe1/graph1.jpg}
		\includegraphics[scale=0.4]{/home/stefan/projects/m_alg/maps/blatt2_img/aufgabe1/graph2.jpg}
		\includegraphics[scale=0.4]{/home/stefan/projects/m_alg/maps/blatt2_img/aufgabe1/graph3.jpg}
		\includegraphics[scale=0.4]{/home/stefan/projects/m_alg/maps/blatt2_img/aufgabe1/graph4.jpg}
		\includegraphics[scale=0.4]{/home/stefan/projects/m_alg/maps/blatt2_img/aufgabe1/graph5.jpg}
		\includegraphics[scale=0.4]{/home/stefan/projects/m_alg/maps/blatt2_img/aufgabe1/graph6.jpg}
		\includegraphics[scale=0.4]{/home/stefan/projects/m_alg/maps/blatt2_img/aufgabe1/graph7.jpg}
		\includegraphics[scale=0.4]{/home/stefan/projects/m_alg/maps/blatt2_img/aufgabe1/graph8.jpg}
		\includegraphics[scale=0.4]{/home/stefan/projects/m_alg/maps/blatt2_img/aufgabe1/graph9.jpg}
		\includegraphics[scale=0.4]{/home/stefan/projects/m_alg/maps/blatt2_img/aufgabe1/graph11.jpg}
		\includegraphics[scale=0.4]{/home/stefan/projects/m_alg/maps/blatt2_img/aufgabe1/graph12.jpg}
		\includegraphics[scale=0.4]{/home/stefan/projects/m_alg/maps/blatt2_img/aufgabe1/graph13.jpg}
		\includegraphics[scale=0.4]{/home/stefan/projects/m_alg/maps/blatt2_img/aufgabe1/graph14.jpg}
		\includegraphics[scale=0.4]{/home/stefan/projects/m_alg/maps/blatt2_img/aufgabe1/graph15.jpg}
		\includegraphics[scale=0.4]{/home/stefan/projects/m_alg/maps/blatt2_img/aufgabe1/graph16.jpg}
		\includegraphics[scale=0.4]{/home/stefan/projects/m_alg/maps/blatt2_img/aufgabe1/graph17.jpg}
		\includegraphics[scale=0.4]{/home/stefan/projects/m_alg/maps/blatt2_img/aufgabe1/graph18.jpg}
		\includegraphics[scale=0.4]{/home/stefan/projects/m_alg/maps/blatt2_img/aufgabe1/graph19.jpg}
		Daraus folgt: $\Rightarrow$\\
		\includegraphics[scale=0.4]{/home/stefan/projects/m_alg/maps/blatt2_img/aufgabe1/graph20.jpg}
		
	
	
	
	
	
	
	
	
	
	
	
	
	
	
	
	
	
	
	
	
	
	
	
	
	
	
	
	
	
	
	\section*{Aufgabe 2}
	\subsection*{(a)}
	
	\includegraphics[scale=0.6]{/home/stefan/projects/m_alg/maps/map_a.png}
	
	\subsection*{(b)}
	\includegraphics[scale=0.4]{/home/stefan/projects/m_alg/maps/b_1.png}
	
	
	\includegraphics[scale=0.4]{/home/stefan/projects/m_alg/maps/b_2}
	
	
	\includegraphics[scale=0.4]{/home/stefan/projects/m_alg/maps/b_3}
	
	
	\includegraphics[scale=0.4]{/home/stefan/projects/m_alg/maps/b_4}
	
	\includegraphics[scale=0.4]{/home/stefan/projects/m_alg/maps/b_5}
	
	
	\includegraphics[scale=0.4]{/home/stefan/projects/m_alg/maps/b_6}
	
	
	\includegraphics[scale=0.4]{/home/stefan/projects/m_alg/maps/b_7}
	
	
	\includegraphics[scale=0.4]{/home/stefan/projects/m_alg/maps/b_9}
	
	%%%%%%%%%%%%
	\includegraphics[scale=0.4]{/home/stefan/projects/m_alg/maps/b_8}
	
	\includegraphics[scale=0.4]{/home/stefan/projects/m_alg/maps/b_10}
	
	\includegraphics[scale=0.4]{/home/stefan/projects/m_alg/maps/b_11}
	
	\includegraphics[scale=0.4]{/home/stefan/projects/m_alg/maps/b_12}
	
	\includegraphics[scale=0.4]{/home/stefan/projects/m_alg/maps/b_13}
	
	\includegraphics[scale=0.4]{/home/stefan/projects/m_alg/maps/b_14}
	
	\subsection*{(c)}
	Die minimalen Kantenkosten betragen insgesamt $65$.\\
	
	\includegraphics[scale=0.4]{/home/stefan/projects/m_alg/maps/b_14}
	












\section*{Aufgabe 3}
\subsection*{(a)}
\includegraphics[scale=0.4]{/home/stefan/projects/m_alg/maps/blatt2_img/aufgabe2/graph1.jpg}
\includegraphics[scale=0.4]{/home/stefan/projects/m_alg/maps/blatt2_img/aufgabe2/graph2.jpg}
\includegraphics[scale=0.4]{/home/stefan/projects/m_alg/maps/blatt2_img/aufgabe2/graph3.jpg}
\includegraphics[scale=0.4]{/home/stefan/projects/m_alg/maps/blatt2_img/aufgabe2/graph4.jpg}
	\includegraphics[scale=0.4]{/home/stefan/projects/m_alg/maps/blatt2_img/aufgabe2/graph5.jpg}
\begin{figure}[ht]
	\includegraphics[scale=0.4]{/home/stefan/projects/m_alg/maps/blatt2_img/aufgabe2/graph6.jpg}
	\caption{Min-Cut: 16}
\end{figure}
\includegraphics[scale=0.4]{/home/stefan/projects/m_alg/maps/blatt2_img/aufgabe2/graph7.jpg}
\includegraphics[scale=0.4]{/home/stefan/projects/m_alg/maps/blatt2_img/aufgabe2/graph8.jpg}
\includegraphics[scale=0.4]{/home/stefan/projects/m_alg/maps/blatt2_img/aufgabe2/graph9.jpg}
\includegraphics[scale=0.4]{/home/stefan/projects/m_alg/maps/blatt2_img/aufgabe2/graph10.jpg}
\includegraphics[scale=0.4]{/home/stefan/projects/m_alg/maps/blatt2_img/aufgabe2/graph11.jpg}
\includegraphics[scale=0.4]{/home/stefan/projects/m_alg/maps/blatt2_img/aufgabe2/graph12.jpg}
\includegraphics[scale=0.4]{/home/stefan/projects/m_alg/maps/blatt2_img/aufgabe2/graph13.jpg}

\subsection*{(b)}
Auch bei negativen Kantengewichten funktioniert der Algorithmus.\\
Da im Algorithmus Kantengewichte aufaddiert werden, und dann die negativen Kantengewichte, einfach abgezogen werden, und der Algorithmus auch so zu einem Ergebnis kommt, das evtl. zwar negativ, also nicht trivial ist.
\section*{Aufgabe 4}
\subsection*{(a)}
\includegraphics*[scale=.5]{/home/stefan/projects/m_alg/maps/4_a}
 

\subsection*{(c)}
1.Bedingung $\Rightarrow q \geq Max(f_i)$\\
$i \in \lbrace 0,1,...,q \rbrace$\\
$Max(f_i) \leq t_i$\\

\includegraphics*[scale=.5]{/home/stefan/projects/m_alg/maps/4_c}
Kapazitaet $t_0 = f_0$(Anzahl Familienmitglieder)
	
\subsection*{(d)}
\includegraphics*[scale=.5]{/home/stefan/projects/m_alg/maps/4_d}
\end{document}