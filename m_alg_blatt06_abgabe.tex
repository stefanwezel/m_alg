% ----------------------- TODO ---------------------------
%Template 
\documentclass[a4paper]{scrartcl}
\usepackage[utf8]{inputenc}
%\usepackage[ngerman]{babel}
\usepackage{geometry,forloop,fancyhdr,fancybox,lastpage}
\usepackage{float}
\usepackage{listings}
\lstset{frame=tb,
	language=Java,
	aboveskip=3mm,
	belowskip=3mm,
	showstringspaces=false,
	columns=flexible,
	basicstyle={\small\ttfamily},
	numbers=left,
	numberstyle=\tiny\color{gray},
	keywordstyle=\color{blue},
	commentstyle=\color{dkgreen},
	stringstyle=\color{mauve},
	breaklines=true,
	breakatwhitespace=true,
	tabsize=3
}
\geometry{a4paper,left=3cm, right=3cm, top=3cm, bottom=3cm}
% Diese Daten müssen pro Blatt angepasst werden:
\newcommand{\NUMBER}{4}
\usepackage{graphicx}
\graphicspath{{/home/stefan/projects/m_alg/}}
\newcommand{\EXERCISES}{5}
% Diese Daten müssen einmalig pro Vorlesung angepasst werden:
\newcommand{\COURSE}{Methoden der Algorithmik}
%\newcommand{\TUTOR}{Benjamin Coban}
\newcommand{\STUDENTA}{Stefan Wezel}
\newcommand{\STUDENTB}{Lukas Günthner}
%\newcommand{\STUDENTC}{Gwent Krause}
\newcommand{\DEADLINE}{\date}
% ----------------------- TODO ---------------------------

%Math
\usepackage{amsmath,amssymb,tabularx}

%Figures
\usepackage{graphicx,tikz,color,float}
\graphicspath{ {home/stefan/picures/} }
\usetikzlibrary{shapes,trees}

%Algorithms
\usepackage[ruled,linesnumbered]{algorithm2e}

%Kopf- und Fußzeile
\pagestyle {fancy}
%\fancyhead[L]{Tutor: \TUTOR}
\fancyhead[C]{\COURSE}
\fancyhead[R]{\today}

\fancyfoot[L]{}
\fancyfoot[C]{}
\fancyfoot[R]{Seite \thepage}

%Formatierung der Überschrift, hier nichts ändern
\def\header#1#2{
	\begin{center}
		{\Large\bf Übungsblatt #1}\\
		{(Abgabetermin #2)}
	\end{center}
}

%Definition der Punktetabelle, hier nichts ändern
\newcounter{punktelistectr}
\newcounter{punkte}
\newcommand{\punkteliste}[2]{%
	\setcounter{punkte}{#2}%
	\addtocounter{punkte}{-#1}%
	\stepcounter{punkte}%<-- also punkte = m-n+1 = Anzahl Spalten[1]
	\begin{center}%
		\begin{tabularx}{\linewidth}[]{@{}*{\thepunkte}{>{\centering\arraybackslash} X|}@{}>{\centering\arraybackslash}X}
			\forloop{punktelistectr}{#1}{\value{punktelistectr} < #2 } %
			{%
				\thepunktelistectr &
			}
			#2 &  $\Sigma$ \\
			\hline
			\forloop{punktelistectr}{#1}{\value{punktelistectr} < #2 } %
			{%
				&
			} &\\
			\forloop{punktelistectr}{#1}{\value{punktelistectr} < #2 } %
			{%
				&
			} &\\
		\end{tabularx}
	\end{center}
}

\begin{document}
	
	\begin{tabularx}{\linewidth}{m{0.2 \linewidth}X}
		\begin{minipage}{\linewidth}
			\STUDENTA\\
			\STUDENTB\\
			%\STUDENTC
		\end{minipage} & \begin{minipage}{\linewidth}
			\punkteliste{1}{\EXERCISES}
		\end{minipage}\\
	\end{tabularx}
	
	%\header{Nr. \NUMBER}{\DEADLINE}
	
	% ----------------------- TODO ---------------------------
	% Hier werden die Aufgaben/Lösungen eingetragen



\section*{Aufgabe 1}
\subsection*{(a)}
Standard Form:\\
$
-2x_1 + x_2 -5x_3 = 2\\
-7x_1 - 3x_2 + 2x_3 - s = 2\\
1 \leq x_2 \leq 19 ;\; x_1, x_3, s \geq 0
$

\subsection*{(b)}
Standard Form:\\
$
y_1 - y_2 - 4x_2 + 3x_3 - s_1 = 42\\
y_1 - y_2 + x_2 - 9x_3 + s_2 = 2\\
x_1 = y_1 - y_2, y_1, y_2, x_2, x_3, s_1, s_2 \geq 0
$



\section*{Aufgabe 2}

\subsection*{(a)}
\begin{figure}[H]
	\includegraphics*[scale = 0.8]{ex06_plot_a}
	\caption{There is no solution to the function.}
\end{figure}



\subsection*{(b)}
\begin{figure}[H]
	\includegraphics*{ex06_plot_b}
	\caption{There is a solution.}
\end{figure}



\subsection*{(c)}
\begin{figure}[H]
	\includegraphics*{ex06_plot_c}
	\caption{There is a solution.}
\end{figure}



\subsection*{(d)}
\begin{figure}[H]
	\includegraphics*{ex06_plot_d}
	\caption{There is a solution.}
\end{figure}








\section*{Aufgabe 3}

Standard Form:\\
$
max(x_1 + 2x_2 + x_3)
$\\
s.t.\\
$
2x_1 + 2x_2 + x_3 + s_1 = 10\\
x_1 + 4x_2 + 2x_3 + s_2 = 11\\
x_1, x_2, x_3, s_1, s_2 \geq 0
$







\end{document}