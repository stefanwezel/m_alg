% ----------------------- TODO ---------------------------
%Template 
\documentclass[a4paper]{scrartcl}
\usepackage[utf8]{inputenc}
%\usepackage[ngerman]{babel}
\usepackage{geometry,forloop,fancyhdr,fancybox,lastpage}
\usepackage{float}
\usepackage{graphics}
\usepackage{color, colortbl}
\usepackage{amsmath}
%\definecolor{name}{system}{definition}
	
\definecolor{Gray}{gray}{0.9}
\definecolor{LightCyan}{rgb}{0.88,1,1}
\usepackage{listings}
\lstset{frame=tb,
	language=Java,
	aboveskip=3mm,
	belowskip=3mm,
	showstringspaces=false,
	columns=flexible,
	basicstyle={\small\ttfamily},
	numbers=left,
	numberstyle=\tiny\color{gray},
	keywordstyle=\color{blue},
	commentstyle=\color{dkgreen},
	stringstyle=\color{mauve},
	breaklines=true,
	breakatwhitespace=true,
	tabsize=3
}
\geometry{a4paper,left=3cm, right=3cm, top=3cm, bottom=3cm}
% Diese Daten müssen pro Blatt angepasst werden:
\newcommand{\NUMBER}{4}
\usepackage{graphicx}
\graphicspath{{/home/stefan/projects/m_alg/}}
\newcommand{\EXERCISES}{4}
% Diese Daten müssen einmalig pro Vorlesung angepasst werden:
\newcommand{\COURSE}{Methoden der Algorithmik}
%\newcommand{\TUTOR}{Benjamin Coban}
\newcommand{\STUDENTA}{Lukas Günthner}
\newcommand{\STUDENTB}{Stefan Wezel}
%\newcommand{\STUDENTC}{Gwent Krause}
\newcommand{\DEADLINE}{\date}
% ----------------------- TODO ---------------------------

%Math
\usepackage{amsmath,amssymb,tabularx}

%Figures
\usepackage{graphicx,tikz,color,float}
\graphicspath{ {home/stefan/picures/} }
\usetikzlibrary{shapes,trees}

%Algorithms
\usepackage[ruled,linesnumbered]{algorithm2e}

%Kopf- und Fußzeile
\pagestyle {fancy}
%\fancyhead[L]{Tutor: \TUTOR}
\fancyhead[C]{\COURSE}
\fancyhead[R]{\today}

\fancyfoot[L]{}
\fancyfoot[C]{}
\fancyfoot[R]{Seite \thepage}

%Formatierung der Überschrift, hier nichts ändern
\def\header#1#2{
	\begin{center}
		{\Large\bf Übungsblatt #1}\\
		{(Abgabetermin #2)}
	\end{center}
}

%Definition der Punktetabelle, hier nichts ändern
\newcounter{punktelistectr}
\newcounter{punkte}
\newcommand{\punkteliste}[2]{%
	\setcounter{punkte}{#2}%
	\addtocounter{punkte}{-#1}%
	\stepcounter{punkte}%<-- also punkte = m-n+1 = Anzahl Spalten[1]
	\begin{center}%
		\begin{tabularx}{\linewidth}[]{@{}*{\thepunkte}{>{\centering\arraybackslash} X|}@{}>{\centering\arraybackslash}X}
			\forloop{punktelistectr}{#1}{\value{punktelistectr} < #2 } %
			{%
				\thepunktelistectr &
			}
			#2 &  $\Sigma$ \\
			\hline
			\forloop{punktelistectr}{#1}{\value{punktelistectr} < #2 } %
			{%
				&
			} &\\
			\forloop{punktelistectr}{#1}{\value{punktelistectr} < #2 } %
			{%
				&
			} &\\
		\end{tabularx}
	\end{center}
}

\begin{document}
	
	\begin{tabularx}{\linewidth}{m{0.2 \linewidth}X}
		\begin{minipage}{\linewidth}
			\STUDENTA\\
			\STUDENTB\\
			%\STUDENTC
		\end{minipage} & \begin{minipage}{\linewidth}
			\punkteliste{1}{\EXERCISES}
		\end{minipage}\\
	\end{tabularx}
	
	%\header{Nr. \NUMBER}{\DEADLINE}
	
	% ----------------------- TODO ---------------------------
	% Hier werden die Aufgaben/Lösungen eingetragen



\section*{Aufgabe 1}
\subsection*{(a)}

\begin{itemize}
\item $max(x_1 - 3x_2 + x_3)$

\item
$
2x_1 - x_2 +5x_3 = -2\\
7x_1 + 3x_2 + 2x_3 \leq -2\\
1 \leq x_2 \leq 19 ;\; x_1, x_3, \geq 0
$
	
	
\item
$
2x_1 - x_2 +5x_3 = -2\\
7x_1 + 3x_2 + 2x_3 - s = -2\\
1 \leq x_2 \leq 19 ;\; x_1, x_3, s \geq 0
$

\item
$
-2x_1 + x_2 -5x_3 = 2\\
7x_1 + 3x_2 + 2x_3 - s = -2\\
1 \leq x_2 \leq 19 ;\; x_1, x_3, s \geq 0
$\\

\item 
Standard Form:\\
$
-2x_1 + x_2 -5x_3 = 2\\
-7x_1 - 3x_2 + 2x_3 - s = 2\\
1 \leq x_2 \leq 19 ;\; x_1, x_3, s \geq 0
$


\end{itemize}
\subsection*{(b)}
\begin{itemize}
\item $max(7x_1 - 2x_2 + x_3)$
\item
$
x_1 - 4x_2 + 3x_3 + \geq 42\\
x_1 + x_2 - 9x_3 \leq 2\\
x_1, x_2, x_3 \geq 0
$
\\
\item
$
y_1 - y_2 - 4x_2 + 3x_3 + \geq 42\\
y_1 - y_2 + x_2 - 9x_3 \leq 2\\
x_1 = y_1 - y_2, y_1, y_2, x_2, x_3 \geq 0
$
\\
\item
$
y_1 - y_2 - 4x_2 + 3x_3 - s_1 = 42\\
y_1 - y_2 + x_2 - 9x_3 \leq 2\\
x_1 = y_1 - y_2, y_1, y_2, x_2, x_3, s_1 \geq 0
$
\\
\item
Standard Form:\\
$
y_1 - y_2 - 4x_2 + 3x_3 - s_1 = 42\\
y_1 - y_2 + x_2 - 9x_3 + s_2 = 2\\
x_1 = y_1 - y_2, y_1, y_2, x_2, x_3, s_1, s_2 \geq 0
$
\end{itemize}


\section*{Aufgabe 2}

\subsection*{(a)}
\begin{figure}[H]
	\includegraphics*[scale = 0.8]{ex06_plot_a}
	\caption{There is no solution to the function.}
\end{figure}



\subsection*{(b)}
\begin{figure}[H]
	\includegraphics*{ex06_plot_b}
	\caption{There is a solution at $x_1 = 4, x_2 = 6, \Rightarrow 36$.}
\end{figure}



\subsection*{(c)}
\begin{figure}[H]
	\includegraphics*{ex06_plot_c}
	\caption{Unbeschraenkt.}
\end{figure}



\subsection*{(d)}
\begin{figure}[H]
	\includegraphics*{ex06_plot_d}
	\caption{There is a solution at $x_1 = 4, x_2 = 6$ and $x_1 = 5, x_2 = 5 \Rightarrow 30$.}
\end{figure}








\section*{Aufgabe 3}

Standard Form:\\
$
max(x_1 + 2x_2 + x_3)
$\\
s.t.\\
$
2x_1 + 2x_2 + x_3 + s_1 = 10\\
x_1 + 4x_2 + 2x_3 + s_2 = 11\\
x_1, x_2, x_3, s_1, s_2 \geq 0
$\\\\
\\
\\
\begin{table}[ht]
\resizebox{\columnwidth}{!}{
\begin{tabular}{|c|c|c|c|c|c|c|c|}
	\hline \rowcolor{lightgray}
	& $x_1$ & $x_2$ & $x_3$ & $s_1$ & $s_2$ & b & t \\ 
	\hline 
	$s_1$ & 2 & 2 & 1 & 1 & 0 & 10 & 5 \\ 
	\hline 
	$s_2$ & 1 & 4 & 2 & 0 & 1 & 11 & $\frac{11}{4}$ \\ 
	\hline 
	p & -1 & -2 & -1 & 0 & 0 & 0 &  \\ 
	\hline 
\end{tabular} 
}
\end{table}
\\
\\
\begin{table}[ht]
	\resizebox{\columnwidth}{!}{
\begin{tabular}{|c|c|c|c|c|c|c|c|}
	\hline \rowcolor{lightgray}
	& $x_1$ & $x_2$ & $x_3$ & $s_1$ & $s_2$ & b & t \\ 
	\hline 
	$s_1$ & $\frac{3}{2}$ & 0 & 1 & 1 & $-\frac{1}{2}$ & $\frac{9}{2}$ & 3 \\ 
	\hline 
	$x_2$ & $\frac{1}{4}$ & 1 & $\frac{1}{2}$ & $\frac{1}{4}$ & $\frac{11}{4}$ & 11 & 11 \\ 
	\hline 
	p & $-\frac{1}{2}$ & 0 & 0 & 0 & $\frac{1}{2}$ & $\frac{11}{2}$ &  \\ 
	\hline 
\end{tabular} 
}
\end{table}
\\
\\
\begin{table}[ht]
	\resizebox{\columnwidth}{!}{
\begin{tabular}{|c|c|c|c|c|c|c|c|}
	\hline \rowcolor{lightgray}
	& $x_1$ & $x_2$ & $x_3$ & $s_1$ & $s_2$ & b & t \\ 
	\hline 
	$x_1$ & 1 & 0 & 0 & $\frac{2}{3}$  & -$\frac{1}{3}$ &3 & \\ 
	\hline 
	$x_2$ & 0 & 1 & $\frac{1}{2}$ & -$\frac{1}{6}$ & $\frac{1}{3}$ & 2 &  \\ 
	\hline 
	p & 0 & 0 & 0 & $\frac{1}{3}$ & $\frac{1}{3}$ & 7 &  \\ 
	\hline 
\end{tabular}
}
\end{table}
\\
\\
\\
\\
\\
\\
\\
\newpage
$x_1 = 3, x_2 = 2, x_3 = 0$ \\
$\Rightarrow$ Maximales Ergebnis: $7$.





\section*{Aufgabe 4}
\subsection*{(a)}
min $9 x_1 15 x_2 24 x_3$\\
s.t.$x_1 \leq 0.3$ hoechstens $30\%$ Zutat 1.\\
$x_2 \leq 0.2$ mindestens $20\%$ Zutat 2.\\
$x_3 \leq 0.2$ mindestens $20\%$ Zutat 3.\\
$250 x_1 + 150 x_2 + 20 x_3 \geq 120$ mindestens 120 Gramm Fett.\\
$300 x_1 + 120 x_2 + 10 x_3 \leq 100$ hoechstens 100 Gramm Zucker.\\
$40 x_1 + 80 x_2 + 20 x_3 \leq 50$ hoechstens 50 Gramm Salz.\\
$250 x_1 + 150 x_2 + 20 x_3 \leq 200$ hoechstens 200 Gramm Fett.\\
\\
min $9 x_1 15 x_2 24 x_3$\\
s.t.$-x_1 \leq -0.3$ hoechstens $30\%$ Zutat 1.\\
$x_2 \leq 0.2$ mindestens $20\%$ Zutat 2.\\
$x_3 \leq 0.2$ mindestens $20\%$ Zutat 3.\\
$250 x_1 + 150 x_2 + 20 x_3 \geq 120$ mindestens 120 Gramm Fett.\\
$-300 x_1 - 120 x_2 - 10 x_3 \geq -100$ hoechstens 100 Gramm Zucker.\\
$-40 x_1 - 80 x_2 - 20 x_3 \geq -50$ hoechstens 50 Gramm Salz.\\
$-250 x_1 - 150 x_2 - 20 x_3 \geq -200$ hoechstens 200 Gramm Fett.\\






\subsection*{(b)}
min.:\\
\begin{align}
	\begin{bmatrix}
	9\\
	15\\
	24
	\end{bmatrix}^T
		\begin{bmatrix}
	x_1\\
	x_2\\
	x_3
	\end{bmatrix}
\end{align}\\
s.t.:\\
\begin{align}
	\begin{bmatrix} 
		-1 & 0 & 0 \\
		0 & 1 & 0 \\
		0 & 0 & 1\\
		250 & 150 & 20\\
		-300 & -120 & -10\\
		-40 & -80 & -20\\
		-250 & -150 &-20
	\end{bmatrix}
			\begin{bmatrix}
	x_1\\
	x_2\\
	x_3
	\end{bmatrix}
	\geq
			\begin{bmatrix}
	-0.3\\
	0.2\\
	0.2\\
	120\\
	-100\\
	-50\\
	-200
	\end{bmatrix}
\end{align}

\subsection*{(c)}
max.:\\
\begin{align}
			\begin{bmatrix}
-0.3 &0.2&0.2&120&-100&-50&-200
\end{bmatrix}
			\begin{bmatrix}
y_1\\
y_2\\
y_3\\
y_3\\
y_5\\
y_6\\
y_7
\end{bmatrix}
\end{align}\\
s.t.:\\
\begin{align}
\begin{bmatrix}
-1 &0&0&250&-300&-40&-250\\
0 &1&0&150&-120&-80&-150\\
0 &0&1&20&-10&-20&-20\\
\end{bmatrix}
			\begin{bmatrix}
y_1\\
y_2\\
y_3\\
y_3\\
y_5\\
y_6\\
y_7
\end{bmatrix}
\leq
			\begin{bmatrix}
9\\
15\\
24\\
\end{bmatrix}
\end{align}
\\
max.:
$-0.3y_1 + 0.2y_2 + 0.2 y_3 + 120 y_4 - 100y_5 - 50 y_6 - 200 y_7$\\
s.t.:\\
$
-y_1 + 250y_4 - 300 y_5 - 40 y_6 - 250 y_7 \leq 9\\
y_2 + 150 y_4 - 120 y_5 - 802 y_6 - 150 y_7 \leq 15\\
y_3 + 20y_4 - 10y_5 - 20 y_6 - 20 y_7 \leq 24
$





\end{document}